% !TEX encoding = UTF-8
% !TEX program = lualatex

\documentclass[a4j,12pt]{ltjtarticle}
\usepackage[paperwidth=20cm, paperheight=15cm, margin=2cm]{geometry} 
\usepackage{luatexja-fontspec}
\usepackage{tocloft}
\usepackage{graphicx}

\setcounter{secnumdepth}{-1}

\setmainjfont[BoldFont=KaiTi]{KaiTi}

\usepackage{hyperref}

\renewcommand\thefootnote{}


%%%%%%%%%%%%%% 目錄 %%%%%%%%%%%%%%%%%%%%%
\makeatletter
\renewcommand{\numberline}[1]{} % 移除目录中的编号
\makeatother
\renewcommand{\cftsecleader}{\cftdotfill{\cftdotsep}}% 为 section 添加点点

%%%%%%%%%%%%%%%%%%%%%%%%%%%%%%%%%%%%%%%%%%



%%%%%%%%%%%%%% 修改字體 %%%%%%%%%%%%%%
\usepackage[utf8]{inputenc}
\usepackage{titlesec}

% 修改 section 標題的字體大小
\titleformat{\section}{\Huge\bfseries}{\thesection}{1em}{}
\usepackage{titletoc}
\titlecontents{section}
[0pt]                  % 左邊距
{\Large}               % 字體設定
{\thecontentslabel\ }  % 條目格式
{}                     % 項目內的標題
{\titlerule*[1pc]{.}\contentspage} % 为 section 添加点点



%%%%%%%%%%%%%%%%%%%%%%%%%%%%%%%%%%%%%


\title{文集}
\author{衡}


\begin{document}
	

	
	
	

	\Huge 野火文集序 \\
		
	\newpage
		
	\tableofcontents
	
	\newpage
	\begin{center}
	
		\begin{flushleft}
			\section{月歌} 
		\end{flushleft}
		
		
		
		\vfill
		\LARGE 	月圓雲襲月如缺 \par
		 		雨陣驚荷雨忽歇 \par
		 		大塊陷人悲喜間 \par
		 		遣人何得求心滅 \par
		\vspace{1cm} % 調節中央對齊
		\vfill
		
		\footnotetext{大塊:況陽春召我以煙景、大塊假我以文章。大塊者,自然也。}
	\newpage
		\begin{flushleft}
			\section{空山}
		\end{flushleft}
		
		\vfill
		\LARGE 	乍寒夜雨濯林空 \par
				靉靆烟雲聞刹鐘 \par
				尋路偶逢歸寺叟 \par
				笑言我復行迷蹤 \par
		\vspace{0.5cm} % 調節中央對齊
		\vfill
		
		\footnotetext{其言尋寺、不如言尋心。刹鐘或可作參考、然烟雲之中,人之本心難求、恒之者更難求。時自省吾身、歸寺叟或即自己。「迷時師度、悟時自度」或言此意。}
		
	\newpage
	
		\begin{flushleft}
			\section{老嫗} 
		\end{flushleft}
		
		\vfill
		\LARGE 	老嫗孤燈下、寒風料峭生 \par
		忘身布衣薄、影蹣足難行 \par
		舊樽復瘦碗、纖手纏囊盈\par
		今夜怎果腹、剩餘糠與粳\par
		我生無功德、腰纏仰家名\par
		唯恨力難及、哀息且憐聲\par
		尋完此街去、燈火髮熒熒\par
		轉去聲俱寂、子夜已三更\par
		\vspace{1.5cm} % 調節中央對齊
		\vfill
		
		
	\newpage
	
		\begin{flushleft}
			\section{不繫舟} 
		\end{flushleft}
		
		\vfill
		\LARGE 	玉鉴纱霧遲\par
		湖光銀漢逝 \par
		孤舟何處系 \par
		且願東風知 \par
		\vspace{1.5cm} % 調節中央對齊
		\vfill
	
	\newpage
	
	\begin{flushleft}
		\section{歸宴中秋月} 
	\end{flushleft}
	
	\vfill
		\LARGE 	夜風料峭岷江邊\par
		子時月明天\par
		光影笙歌弦\par 怎教人知\par
		孤孑皓月圓\par
		我欲乘風上瓊樓\par
		殘燈夜半後\par
		世事大夢方得悟\par 風清雲淡\par
		同天畫玉壺\par
		\vspace{1.5cm} % 調節中央對齊
	\vfill
		
	
	\newpage
	
	\begin{flushleft}
		\section{世風} 
	\end{flushleft}
	
	\vfill
	\LARGE 	致君堯舜難任身\par
	彈冠相比意何存 \par
	花開花落風清月 \par
	客來客往亦行人 \par
	\vspace{1.5cm} % 調節中央對齊
	\vfill
	
	\newpage
	
	\begin{flushleft}
		\section{安順花舞} 
	\end{flushleft}	
	
	
	\vfill
	\LARGE 	飛花落葉殘夏暮\par
	但見乘風亂紅舞 \par
	流水相携天邊去 \par
	花心不住人心住 
	\vfill
	
	\footnotetext{飛花落葉:佛語、謂獨覺乘之人、獨入於林中、見飛花落葉、感知世間無常之理、因而開悟。住:佛語、住相。此詩乃讀「日本文化史」偶憶去年殘夏妙景、當時難以言说、遂作一詩以志。}
	
	\newpage
	\vspace{3cm}
	\vfill
	\begin{figure}[h!]
		\centering
		\includegraphics[angle=270, width=0.8\textwidth]{img/anshun.png}
		\caption{黔東南安順(自攝)}
		\label{fig:zhenwu}
	\end{figure}
	\vfill
	
	\newpage
	
	\begin{flushleft}
		\section{竹溪} 
	\end{flushleft}	
	
	\vfill
	\LARGE 	竹澗鎖春光\par
	拂來緑玉觴 \par
	無舟聞翠浪 \par
	墙下搗衣娘 \par
	\vfill
	
	\footnotetext{緑玉觴:水上之竹葉如流觴、又似有玉之黄緑混濁之色、稱之綠玉觴。此詩乃自錦城還鄉偶過之、雖已暮春曉夏、然天朗氣清、溪仍似有柔春之景。竹溪者嘉州之溪一也、曩者余日過之、奈何匆匆、無暇著詩歌之。今偶得興、遂作一詩以志鄉景。}
	
	
	
	\newpage
	
	\begin{flushleft}
		\section{記夢} 
	\end{flushleft}	
	
	\vfill
		\LARGE 夜半幽人沐月登 \par
		荒樓衰草遍天横 \par
		望如落葉影飛下 \par
		應斷恨腸却斷生 
		\vspace{2cm} % 調節中央對齊
	\vfill
	
	\newpage
	
	\begin{flushleft}
		\section{咏真武山} 
	\end{flushleft}	
	
	
	\vfill
		\LARGE 瓊樓争秀鬥奇松 \par
		真武歸山忘紫宫 \par
		竹椅閑言茶休盡 \par
		人間亦在此山中 
		\vspace{1cm} % 調節中央對齊
	\vfill
	
	\footnotetext{游真武山而後覺蒼翠陰翳、宛遁入仙境、故作詩歌之。}
	
	\newpage
	\vspace{3cm}
	\vfill
	\begin{figure}[h!]
		\centering
		\includegraphics[angle=90, width=0.8\textwidth]{img/zhenwumnt.png}
		\caption{宜賓真武山(自攝)}
		\label{fig:zhenwu}
	\end{figure}
	\vfill
	\newpage
	
	\begin{flushleft}
		\section{别蓉} 
	\end{flushleft}	
	
	\vfill
		\LARGE 錦城何日回 \par
		辭日粼波暉 \par
		蓉水休傷别 \par
		岷江偕我歸
		\vspace{2cm} % 調節中央對齊
	\vfill
	
	\newpage
	
	\begin{flushleft}
		\section{如夢令} 
	\end{flushleft}	
	
	
	\vfill
		\LARGE 新櫻枝展幾旋 \par
		殘寒紅亂行阡 \par
		閑獨步花路 \par
		稀星孤月迷煙 \par
		無眠  無眠 \par
		伊人又奏幽弦
		\vspace{2cm} % 調節中央對齊
	\vfill
	
	
	\newpage
	
	\begin{flushleft}
		\section{獨吟} 
	\end{flushleft}	
	
	\vfill
		\LARGE 酒苦今宵盡 \par
		夢空明日新 \par
		閑鴉無問暮 \par
		偷活不知春 
		\vspace{2cm} % 調節中央對齊
	\vfill
	
	\newpage
	\vfill
	\begin{figure}[h!]
		\centering
		\includegraphics[angle=90, width=0.8\textwidth]{img/crow.jpg}
		\caption{日暮閑鴉(代爾夫特)}
		\label{fig:crow}
	\end{figure}
	\vfill
	\newpage
	
	\begin{flushleft}
		\section{寒秋} 
	\end{flushleft}	
	
	\vfill
		\LARGE 殘年霜降幾寒雨 \par
		鴉雀晚昏啼亂飛 \par
		秋盡黄花才報信 \par
		一鄉明月亦希微 
		\vspace{2cm} % 調節中央對齊
	\vfill
	\newpage
	
	\begin{flushleft}
		\section{一剪梅 耶誕} 
	\end{flushleft}	
	
	
	\vfill
	\LARGE 映月黄花散凍灘 \par
	秋也飄殘  葉也凋殘 \par
	華燈元彩路清閑 \par
	燈影未闌  人影先闌 \par
	吹雪初翻驚孤鸞 \par
	亦似形單 亦似心單 \par
	薄村冷夜計春還 \par
	人宿温寒 我宿孤寒 \par
	
	
	\vspace{2cm} % 調節中央對齊
	\vfill
	
	\footnotetext{如題、耶誕節有感。}
	
	\newpage
	
	\begin{flushleft}
		\section{花非花 遙夜} 
	\end{flushleft}	
	
	\vfill
	\LARGE 風驚簾    雨侵夢\par
	老葉殘    沉潭凍\par
	天光促促幾多時 \par
	遥夜長長誰與共 \par

	\vspace{0.5cm} % 調節中央對齊
	\vfill
	
	\footnotetext{本詞受少游之如夢令 遙夜沉沉如水之啓發、荷蘭低地常作狂風、故深夜有感而發。}
	
	\newpage 
	
	
	\begin{flushleft}
		\section{如夢令 憶昔} 
	\end{flushleft}	
	
	\vfill
	\LARGE 泣笛閨中幾轉\par
	薄暮閑鐘催晚\par
	仍記與伊游\par 
	粤水蜀山恨短 \par
	如幻    如幻\par
	人易星移誰算\par
	\vspace{1.5cm} % 調節中央對齊
	\vfill
	
	
	\newpage

	\begin{flushleft}
		\section{咏金華天寧寺}
	\end{flushleft}

	\vfill
		\LARGE 孤丘深寄萬家間\par
		寶殿莊嚴倚峭前\par
		香盡僧離如來去\par
		寺空空道非空緣\par
		\vspace{1.5cm} % 調節中央對齊
	\vfill

	\footnotetext{二十三年訪金華天寧寺、雄踞隴丘、遙面婺水。祥符初成、延祐重修。而今無人問釋、僧去廟空、唯剩明月清風與寶殿。故生荒涼、世事沉浮、故作一首嘆之。}
	
	\newpage
	\vfill
	\begin{figure}[h!]
		\centering
		\includegraphics[angle=90, width=0.8\textwidth]{img/tianningsi.jpeg}
		\caption{浙江金華天寧寺}
		\label{fig:tianningsi}
	\end{figure}
	\vfill


	\begin{flushleft}
		\section{次年聖誕}
	\end{flushleft}

	\vfill
		\LARGE 胡琴華舞卻冬寒\par
		陌路游人照膽肝\par
		美酒千樽豈能斷\par
		勸君進酒夜未闌\par
		\vspace{1.5cm} % 調節中央對齊
	\vfill

	\footnotetext{二十四年聖誕、因緣訪燈火會記、尾句未者平仄顛倒、以期更正}
	

	\end{center}


	
\end{document}
