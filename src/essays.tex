% !TEX encoding = UTF-8
% !TEX program = lualatex

\documentclass[a4j,12pt]{ltjtarticle}
\usepackage[paperwidth=20cm, paperheight=15cm, margin=2cm]{geometry} 
\usepackage{luatexja-fontspec}
\usepackage{tocloft}
\usepackage{graphicx}

\setcounter{secnumdepth}{-1}

\setmainjfont[BoldFont=KaiTi]{KaiTi}

\usepackage{hyperref}

\renewcommand\thefootnote{}


%%%%%%%%%%%%%% 目錄 %%%%%%%%%%%%%%%%%%%%%
\makeatletter
\renewcommand{\numberline}[1]{} % 移除目录中的编号
\makeatother
\renewcommand{\cftsecleader}{\cftdotfill{\cftdotsep}}% 为 section 添加点点

%%%%%%%%%%%%%%%%%%%%%%%%%%%%%%%%%%%%%%%%%%



%%%%%%%%%%%%%% 修改字體 %%%%%%%%%%%%%%
\usepackage[utf8]{inputenc}
\usepackage{titlesec}

% 修改 section 標題的字體大小
\titleformat{\section}{\Huge\bfseries}{\thesection}{1em}{}
\usepackage{titletoc}
\titlecontents{section}
[0pt]                  % 左邊距
{\Large}               % 字體設定
{\thecontentslabel\ }  % 條目格式
{}                     % 項目內的標題
{\titlerule*[1pc]{.}\contentspage} % 为 section 添加点点

\titlecontents{subsection}
[0pt]                  % 左邊距
{\large}               % 字體設定
{\thecontentslabel\ }  % 條目格式
{}                     % 項目內的標題
{\titlerule*[1pc]{.}\contentspage} % 为 section 添加点点



\begin{document}
	

	
	
	

	\Huge{野火文集序} 
		
	\newpage

    \tableofcontents
	
	\newpage
    %%%%%% 金剛經 %%%%%%%
    \section{金剛經} 

    \normalsize 金剛經者 禪宗經典之一也 歷代研習者無數 本書依循後秦之羅什之譯本 照之以五家之言 

    以宗教理法視之 佛陀言法誠正善 然極樂世界亦誠修遠也 故退卻其道流 免閑于修行耳

    以哲學理論察之 其言也大有裨益 啓發甚深 糾其明義 亦為處世之學 所以流芳西國 奉爲圭臬 冠之禪學
    
    我心中無佛 且無修行 故難踐之 且抒己所感 達己鄙見 

    全經共三十二分 題材廣闊 故唯幾文章 不甚達其所意
    
    佛陀之言不可謂深 然以回文曡詞飾之 故難誤句讀 或述之宗教術語 故諱莫如深 望鄙人之文字 披云見日 推之以理 見之以哲

    \subsection{其一 降伏其心}
    
    \subsection{其二 無得無說}
    

\end{document}