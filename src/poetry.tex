% !TEX encoding = UTF-8
% !TEX program = lualatex

\documentclass[a4j,12pt]{ltjtarticle}
\usepackage[paperwidth=20cm, paperheight=15cm, margin=2cm]{geometry}
\usepackage{luatexja-fontspec}
\usepackage{luatexja-ruby}
\usepackage{graphicx}
\usepackage{titletoc}
\usepackage{titlesec}
\usepackage{zhnumber}
\usepackage{hyperref} % 放最后

\renewcommand{\contentsname}{目錄}  % 想要的标题
\renewcommand{\figurename}{圖} 
\renewcommand{\thefigure}{\zhnumber{\arabic{figure}}} 
\renewcommand{\thefootnote}{\zhnumber{\arabic{footnote}}}

\setmainjfont[BoldFont=STZHONGS]{STSONG}
\newjfontfamily\kaiti{KaiTi}   % Windows 下字体名通常是 “KaiTi”
\setcounter{secnumdepth}{-1} % 让正文不编号(不影响 ToC 分页)

% 正文里的 section 样式
\titleformat{\section}{\Huge\bfseries}{\thesection}{1em}{}

% 頁碼



% 目录里 section 项的样式(大号字、点线、页码)
\titlecontents{section}
  [5pt]                % 左缩进
  {\Large}             % 字体
  {}                   % 标签(留空 = 不显示编号)
  {}                   % 标题前缀
%   {\titlerule*[1pc]{.}\contentspage} % 右侧点线 + 页码
  {\titlerule*[1pc]{}\large\kaiti\contentspage}

% 脚注
\renewcommand{\footnotesize}{\kaiti\small}
\makeatletter
\renewcommand{\@makefnmark}{{\kaiti\textsuperscript{\@thefnmark}}}
\makeatother

%图片
\usepackage{caption}
\DeclareCaptionFont{kaiti}{\kaiti}
\captionsetup{font=kaiti}


%%%%%%%%%%%%%%%%%%%%%%%%%%%%%%%%%%%%%


\begin{document}
\pagenumbering{arabic}
\renewcommand{\thepage}{{\kaiti\zhnumber{\arabic{page}}}}
\setcounter{page}{1}

	\Huge 野火诗集序 \\

	\Large
		野火者東洋言訓也野火先生者本嘉州人
	也及至弱冠遂就成都長宿之處嘗閑起詩
	興尋寺涉水快意其間後徙蘭國輾轉研\ruby{縣}{\footnotemark}\footnotetext{研縣者、通語呼台夫特也。然自以爲其名甚異、捨義逐音、遂自詡之研縣、取荷語鑽研之意義。}
	營營恂恂悶悶終日憂光陰迅速欲著文字
	又恐多忙無暇相顧遂訴諸詩賦且狀舊故
	夫世文集浩浩宋詞唐詩著盡風雅然本集
	不及其半分既然何爲屬文曩有唐科舉詩
	賦取士後有宋詞盛結貴攀援然今者何也
	蓋抒己志也乃人情流變甚於山雨傷春凄
	秋朝喜暮悲此情一瞬縱逝不復此時縱之
	更待何時哉夫命且蜉蝣鴻尚踏雪人何以
	堪况凝文結字而後品誦亦似他言蓋時過
	情遷縱景同而情异也

	及此且憑此集略言我情夫詩唯志己悲喜
	也夫句唯與己沉吟也且取樂其間悠悠無
	縛哉時二十四年三月
	
		
	\newpage
		
	\tableofcontents
	
	\newpage
	\begin{center}
	
		\begin{flushleft}
			\section{月歌} 
		\end{flushleft}
		
		
		
		\vfill
		\LARGE 	月圓雲襲月如缺 \par
		 		雨陣驚荷雨忽歇 \par
		 		大塊陷人悲喜間 \par
		 		遣人何得求心滅 \par
		\vspace{1cm} % 調節中央對齊
		\vfill
		
		\footnotetext{大塊:況陽春召我以煙景、大塊假我以文章。大塊者,自然也。}
	\newpage
		\begin{flushleft}
			\section{空山}
		\end{flushleft}
		
		\vfill
		\LARGE 	乍寒夜雨濯林空 \par
				靉靆烟雲聞刹鐘 \par
				尋路偶逢歸寺叟 \par
				笑言我復行迷蹤 \par
		\vspace{0.5cm} % 調節中央對齊
		\vfill
		
		\footnotetext{其言尋寺、不如言尋心。刹鐘或可作參考、然烟雲之中,人之本心難求、恒之者更難求。時自省吾身、歸寺叟或即自己。「迷時師度、悟時自度」或言此意。}
		
	\newpage
	
		\begin{flushleft}
			\section{老嫗} 
		\end{flushleft}
		
		\vfill
		\LARGE 	老嫗孤燈下、寒風料峭生 \par
		忘身布衣薄、影蹣足難行 \par
		舊樽復瘦碗、纖手纏囊盈\par
		今夜怎果腹、剩餘糠與粳\par
		我生無功德、腰纏仰家名\par
		唯恨力難及、哀息且憐聲\par
		尋完此街去、燈火髮熒熒\par
		轉去聲俱寂、子夜已三更\par
		\vspace{1.5cm} % 調節中央對齊
		\vfill
		
		
	\newpage
	
		\begin{flushleft}
			\section{不繫舟} 
		\end{flushleft}
		
		\vfill
		\LARGE 	玉鉴纱霧遲\par
		湖光銀漢逝 \par
		孤舟何處系 \par
		且願東風知 \par
		\vspace{1.5cm} % 調節中央對齊
		\vfill
	
	\newpage
	
	\begin{flushleft}
		\section{歸宴中秋月} 
	\end{flushleft}
	
	\vfill
		\LARGE 	夜風料峭岷江邊\par
		子時月明天\par
		光影笙歌弦\par 怎教人知\par
		孤孑皓月圓\par
		我欲乘風上瓊樓\par
		殘燈夜半後\par
		世事大夢方得悟\par 風清雲淡\par
		同天畫玉壺\par
		\vspace{1.5cm} % 調節中央對齊
	\vfill
		
	
	\newpage
	
	\begin{flushleft}
		\section{世風} 
	\end{flushleft}
	
	\vfill
	\LARGE 	致君堯舜難任身\par
	彈冠相比意何存 \par
	花開花落風清月 \par
	客來客往亦行人 \par
	\vspace{1.5cm} % 調節中央對齊
	\vfill
	
	\newpage
	
	\begin{flushleft}
		\section{安順花舞} 
	\end{flushleft}	
	
	
	\vfill
	\LARGE 	飛花落葉殘夏暮\par
	但見乘風亂紅舞 \par
	流水相携天邊去 \par
	花心不住人心住 
	\vfill
	
	\footnotetext{飛花落葉:佛語、謂獨覺乘之人、獨入於林中、見飛花落葉、感知世間無常之理、因而開悟。住:佛語、住相。此詩乃讀「日本文化史」偶憶去年殘夏妙景、當時難以言说、遂作一詩以志。}
	
	\newpage
	\vspace{3cm}
	\vfill
	\begin{figure}[h!]
		\centering
		\includegraphics[angle=270, width=0.8\textwidth]{../img/anshun.png}
		\caption{黔東南安順(自攝)}
		\label{fig:zhenwu}
	\end{figure}
	\vfill
	
	\newpage
	
	\begin{flushleft}
		\section{竹溪} 
	\end{flushleft}	
	
	\vfill
	\LARGE 	竹澗鎖春光\par
	拂來緑玉觴 \par
	無舟聞翠浪 \par
	墙下搗衣娘 \par
	\vfill
	
	\footnotetext{緑玉觴:水上之竹葉如流觴、又似有玉之黄緑混濁之色、稱之綠玉觴。此詩乃自錦城還鄉偶過之、雖已暮春曉夏、然天朗氣清、溪仍似有柔春之景。竹溪者嘉州之溪一也、曩者余日過之、奈何匆匆、無暇著詩歌之。今偶得興、遂作一詩以志鄉景。}
	
	
	
	\newpage
	
	\begin{flushleft}
		\section{記夢} 
	\end{flushleft}	
	
	\vfill
		\LARGE 夜半幽人沐月登 \par
		荒樓衰草遍天横 \par
		望如落葉影飛下 \par
		應斷恨腸却斷生 
		\vspace{2cm} % 調節中央對齊
	\vfill
	
	\newpage
	
	\begin{flushleft}
		\section{咏真武山} 
	\end{flushleft}	
	
	
	\vfill
		\LARGE 瓊樓争秀鬥奇松 \par
		真武歸山忘紫宫 \par
		竹椅閑言茶休盡 \par
		人間亦在此山中 
		\vspace{1cm} % 調節中央對齊
	\vfill
	
	\footnotetext{游真武山而後覺蒼翠陰翳、宛遁入仙境、故作詩歌之。}
	
	\newpage
	\vspace{3cm}
	\vfill
	\begin{figure}[h!]
		\centering
		\includegraphics[angle=90, width=0.8\textwidth]{../img/zhenwumnt.png}
		\caption{宜賓真武山(自攝)}
		\label{fig:zhenwu}
	\end{figure}
	\vfill
	\newpage
	
	\begin{flushleft}
		\section{别蓉} 
	\end{flushleft}	
	
	\vfill
		\LARGE 錦城何日回 \par
		辭日粼波暉 \par
		蓉水休傷别 \par
		岷江偕我歸
		\vspace{2cm} % 調節中央對齊
	\vfill
	
	\newpage
	
	\begin{flushleft}
		\section{如夢令} 
	\end{flushleft}	
	
	
	\vfill
		\LARGE 新櫻枝展幾旋 \par
		殘寒紅亂行阡 \par
		閑獨步花路 \par
		稀星孤月迷煙 \par
		無眠  無眠 \par
		伊人又奏幽弦
		\vspace{2cm} % 調節中央對齊
	\vfill
	
	
	\newpage
	
	\begin{flushleft}
		\section{獨吟} 
	\end{flushleft}	
	
	\vfill
		\LARGE 酒苦今宵盡 \par
		夢空明日新 \par
		閑鴉無問暮 \par
		偷活不知春 
		\vspace{2cm} % 調節中央對齊
	\vfill
	
	\newpage
	\vfill
	\begin{figure}[h!]
		\centering
		\includegraphics[angle=90, width=0.8\textwidth]{../img/crow.jpg}
		\caption{日暮閑鴉(代爾夫特)}
		\label{fig:crow}
	\end{figure}
	\vfill
	\newpage
	
	\begin{flushleft}
		\section{寒秋} 
	\end{flushleft}	
	
	\vfill
		\LARGE 殘年霜降幾寒雨 \par
		鴉雀晚昏啼亂飛 \par
		秋盡黄花才報信 \par
		一鄉明月亦希微 
		\vspace{2cm} % 調節中央對齊
	\vfill
	\newpage
	
	\begin{flushleft}
		\section{一剪梅 耶誕} 
	\end{flushleft}	
	
	
	\vfill
	\LARGE 映月黄花散凍灘 \par
	秋也飄殘  葉也凋殘 \par
	華燈元彩路清閑 \par
	燈影未闌  人影先闌 \par
	吹雪初翻驚孤鸞 \par
	亦似形單 亦似心單 \par
	薄村冷夜計春還 \par
	人宿温寒 我宿孤寒 \par
	
	
	\vspace{2cm} % 調節中央對齊
	\vfill
	
	\footnotetext{如題、耶誕節有感。}
	
	\newpage
	
	\begin{flushleft}
		\section{花非花 遙夜} 
	\end{flushleft}	
	
	\vfill
	\LARGE 風驚簾    雨侵夢\par
	老葉殘    沉潭凍\par
	天光促促幾多時 \par
	遥夜長長誰與共 \par

	\vspace{0.5cm} % 調節中央對齊
	\vfill
	
	\footnotetext{本詞受少游之如夢令 遙夜沉沉如水之啓發、荷蘭低地常作狂風、故深夜有感而發。}
	
	\newpage 

	\begin{flushleft}
		\section{研縣新寺} 
	\end{flushleft}	

	\vfill
	\LARGE 高椽秀塔暗天光\par
	拱月攢星隽玉章\par
	笙弄百年悲喜事\par
	地碑忠骨遺苍凉\par
	\vspace{1.5cm} % 調節中央對齊
	\vfill
	
	\newpage

	\begin{flushleft}
		\section{無題} 
	\end{flushleft}	

	\vfill
	\LARGE 挑燈案上聽寒雨\par
	逸話閑茶送古香\par
	西海龍王知我興\par
	贈來夜半一分凉\par
	\vspace{1.5cm} % 調節中央對齊
	\vfill

	\newpage


	\begin{flushleft}
		\section{無題} 
	\end{flushleft}	


	\vfill
	\LARGE 晚安唯兩言\par
	語罷音塵絕\par
	杳杳暮寒天\par
	貓生無異別\par
	\vspace{1.5cm} % 調節中央對齊
	\vfill

	\newpage

	\begin{flushleft}
		\section{寡貓歌} 
	\end{flushleft}	


	\vfill
	\LARGE 夕夕寡猫泣\par
	朝朝淚遺痕\par
	迢迢前路遠\par
	哽哽復無言\par
	\vspace{1.5cm} % 調節中央對齊
	\vfill

	\newpage

	\begin{flushleft}
		\section{蘭國興廢事} 
	\end{flushleft}	

	\vfill
	\LARGE 競彩揚燈臺攤立\par
	古磬瓊樓人影翕\par
	臨風持酒暮忘歸\par
	鐘樂杳鳴催客入\par
	蘭國興亡多少集\par
	酒客遊童誰肯拾\par
	殘牆疏牖屬英名\par
	足下靈堂何者泣\par
	\vspace{1.5cm} % 調節中央對齊
	\vfill

	\footnotetext{蘭國者、現名荷蘭、和譯和蘭也。是日研縣趕場、蘭國古城皆中之以廣場、食肆酒家、橫次其側。其間塔刹巍峨、七寶仙境、勝須彌聖山也。信步殿内、或國之忠烈、或顯貴達官、墓碑悉數足下。嘆已為陳跡、遂予作文。}
	\newpage
	
	\begin{flushleft}
		\section{如夢令 憶昔} 
	\end{flushleft}	
	
	\vfill
	\LARGE 泣笛閨中幾轉\par
	薄暮閑鐘催晚\par
	仍記與伊游\par 
	粤水蜀山恨短 \par
	如幻    如幻\par
	人易星移誰算\par
	\vspace{1.5cm} % 調節中央對齊
	\vfill
	
	
	\newpage

	\begin{flushleft}
		\section{咏金華天寧寺}
	\end{flushleft}

	\vfill
		\LARGE 孤丘深寄萬家間\par
		寶殿莊嚴倚峭前\par
		香盡僧離如來去\par
		寺空空道非空緣\par
		\vspace{1.5cm} % 調節中央對齊
	\vfill

	\footnotetext{二十三年訪金華天寧寺、雄踞隴丘、遙面婺水。祥符初成、延祐重修。而今無人問釋、僧去廟空、唯剩明月清風與寶殿。故生荒涼、世事沉浮、故作一首嘆之。}
	
	\newpage
	\vfill
	\begin{figure}[h!]
		\centering
		\includegraphics[angle=90, width=0.8\textwidth]{../img/tianningsi.jpeg}
		\caption{浙江金華天寧寺}
		\label{fig:tianningsi}
	\end{figure}
	\vfill


	\begin{flushleft}
		\section{次年聖誕}
	\end{flushleft}

	\vfill
		\LARGE 胡琴華舞卻冬寒\par
		陌路游人照膽肝\par
		美酒千樽豈能斷\par
		勸君進酒夜未闌\par
		\vspace{1.5cm} % 調節中央對齊
	\vfill

	\footnotetext{二十四年聖誕、因緣訪燈火會記、尾句未者平仄顛倒、以期更正}
	\newpage
	\begin{flushleft}
		\section{記蜀夏}
	\end{flushleft}
	
	\vfill
	\LARGE 三更雷雨朝來歇\par

	初曉青天紅日烈\par 

	急雨襲城晏倏晴\par

	幽灯人定河明滅\par
	\vspace{1.5cm} % 調節中央對齊
	\vfill

	\footnotetext{二十四年秋、偶憶幼時之蜀地氣象、驀然覺之竟渾然不似他國異鄉、甚殊之、遂以鄉韻偶吟小詩記之。}




	\newpage
	\begin{flushleft}
		\section{望海牙暮煙有感}
	\end{flushleft}
	
	\vfill
	\LARGE 餘曛掩暮煙\par 
		海晏萬鱗連\par 
		眛谷欲邀酒\par 
		來期難問年\par 
	\vspace{1.5cm} % 調節中央對齊
	\vfill

	\footnotetext{二十四年夏、同窗故友邀余以酒、遂往海牙之灘塗。適日薄、誠奇景也。然勝景依舊、酒肆猶沽、故知難再。}

	\newpage
	\vfill
	\begin{figure}[h!]
		\centering
		\includegraphics[angle=90, width=0.8\textwidth]{../img/hague.jpg}

		\caption{海牙勝景 漁燈暮紗}
		\label{fig:hague}
	\end{figure}
	\vfill



	\newpage
	\begin{flushleft}
		\section{青玉案 憶廣州}
	\end{flushleft}
	
	\vfill
	\LARGE 三年别夢羊城路\par 恨當日 人匆去 \par 不識君名難有素 \par 六榕鈡磬 琶洲津渡 \par 欲往君行處 \par \vfill
	是時依約花都赴\par 對君言 衷腸句 \par 晝短情思難盡訴 \par 經年今去 何期再遇 \par 儘付他鄉雨
	\vspace{1.5cm} % 調節中央對齊
	\vfill

	\newpage
	\begin{flushleft}
		\section{漁歌子}
	\end{flushleft}

	\vfill
	\LARGE 葡國景 華賭戲\par  
	燈肆霓虹無寐\par
	紙迷金碎勝飛花\par
	且舞且歌裝醉\par
	\vspace{1.5cm} % 調節中央對齊
	\vfill

	\newpage
	\vfill
	\begin{figure}[h!]
		\centering
		\includegraphics[angle=90, width=0.8\textwidth]{../img/aomen.png}
		\caption{澳門市肆}
		\label{fig:aomen}
	\end{figure}
	\vfill

	\vfill
	\begin{figure}[h!]
		\centering
		\includegraphics[angle=90, width=0.8\textwidth]{../img/aomen-2.png}
		\caption{聖保祿堂望瓊樓}
		\label{fig:aomen2}
	\end{figure}
	\vfill

	\end{center}


	
\end{document}
